\documentclass[12pt, a4paper]{ctexart}

\usepackage{amsmath, amsthm, amssymb, graphicx,float}
\usepackage[bookmarks=true, colorlinks, citecolor=blue, linkcolor=black]{hyperref}

\title{数字逻辑项目文档:GenshinKitchen}
\author{黄政东\\祝超\\何俞均}
\date{}

\begin{document}
\maketitle

\newpage
\section{团队分工}
\subsection{团队分工}
\begin{itemize}
    \item 贡献比\texttt{1:1:1}
    \item  详细工作安排见下表。
\end{itemize}
\begin{table}[h]
	\centering
	\begin{tabular}{|c|c|}
		\hline
		~      & project 顶层模块的处理(manualTop, ScriptTop 以及 DemoTop) \\
		\cline{2-2}
		~      & 实现接收并用LED显示来自客户端的四个反馈信号                 \\
		\cline{2-2}
		黄政东 & 所有模块的代码规范性检查                                    \\
		\cline{2-2}
		~      & 实现在手动模式/自动(脚本)模式之间进行切换                   \\
		\cline{2-2}
		~      & 实现脚本单步调试以及手动模式/自动(脚本)模式之间的切换         \\
		\cline{2-2}
		~      & 部分文档攥写工作                                            \\
		\hline

		~      & project 手动模式的合法性检查                                \\
		\cline{2-2}
		~      & project 脚本模式中jump,wait语句相关模块执行的处理          \\
		\cline{2-2}
		祝超   & 各种需要的测试脚本的准备,高效的脚本设计                    \\
		\cline{2-2}
		~      & loadingLamp 模块的编写                                      \\
		\cline{2-2}
		~      & 部分文档攥写工作                                            \\

		\hline
		~      & project 手动模式的按键到机器码部分(buttonDecoder模块)     \\
		\cline{2-2}
		~      & project 脚本模式中Action、Game State Instruction部分的处理  \\
		\cline{2-2}
		何俞均 & 使用七段数码管显示信息                   \\
		\cline{2-2}
		~      & 错误脚本状态自动处理                                        \\
		\cline{2-2}
		~      & 部分文档攥写工作                                            \\
		\hline
	\end{tabular}
	\caption{详细工作安排}
\end{table}

\subsection{开发计划日程安排和实施情况}
\begin{itemize}
\item \texttt{11.25}: 开会讨论设计大致模块。
                决定设计\texttt{DesignedTop, ManualTop, ScriptTop,TargetRegister}。
                并先完成\texttt{DesignedTop, ManualTop, TargetRegister}。
                
\item \texttt{12.2}: \texttt{DesignedTop, TargetRegister} 完成。
                设计 \texttt{ManualTop} 中包含的 \texttt{TargetStateMachine, TargetStateEncoder, GameStateEncoder, OperationEncoder, ManualFliter}。
                
\item \texttt{12.10}: \texttt{ManualTop} 完成
                设计 \texttt{ScriptTop} 中的 \texttt{GameStateScriptHandler, ActionScriptHandler, JumpScriptHandler, WaitScriptHandler}。
                
\item \texttt{12.17}: \texttt{ScriptTop} 完成
                准备 \texttt{Bonus} 中的 \texttt{ScriptFixer} 以及快速脚本。
                
\item \texttt{12.23}: \texttt{ScriptTop} 中出现 \texttt{bug},没修好。
\item \texttt{12.26}: \texttt{ScriptTop} 修好,快速脚本完成。
                        \texttt{ScriptFixer} 拼不上去。
\item \texttt{12.27}: \texttt{ScriptFixer} 修不好,决定放弃。
                      决定整理代码于 \texttt{12.29}答辩。
\item \texttt{12.28}:加入走马灯,整理代码完成。
\end{itemize}
\section{系统功能列表}
\subsection{手动模式}
\begin{enumerate}
	\item 通过按钮,玩家可以进行实现:
        \begin{itemize}
            \item \texttt{game start/end}:分别对应了拨码\texttt{0/1}。
            \item \texttt{get}:右侧左边按钮。
            \item \texttt{put}:右侧下边按钮。
            \item \texttt{move}:右侧中间上面按钮。
            \item \texttt{interact}:右侧中间按钮。
            \item \texttt{throw}:右侧右边按钮。
            \item \texttt{change target machine}:2号拨码开关为顺时针移动一格,3号为逆时针移动一格。
        \end{itemize}
	\item 对于开始和结束游戏:
        \begin{itemize}
            \item 开始游戏:拨动拨码0,可以开始游戏。玩家可以通过按钮进行各种操作。如果当前正处于脚本模式,则会直接进入手动模式。
            \item 结束游戏:先拨动拨码1,再回拨拨码0,可以结束游戏。这是为了防止拨码接触不良以及玩游戏时误启动\texttt{game end},设计的“双保险”。
        \end{itemize}
	\item 可以自动过滤掉非法的操作:
        \begin{itemize}
            \item 开发板能够阻止移动时(玩家未在机器跟前时)的非法交互。
            \item 开发板能够阻止不合理的存取物品交互。
            \item 开发板能够阻止不合理的投掷食材操作(只可以投掷到桌子/垃圾桶)。
            \item 开发板能保证操作信号是\texttt{One-Hot}编码的。
        \end{itemize}
\end{enumerate}

\subsection{脚本模式}
\begin{enumerate}
	\item 通过向上波动拨码0可以实现手动模式与脚本模式的切换。
	\item 通过输入脚本,可以实现\texttt{get,put,interact,throw,wait,waituntil, jump, jumpif}等操作。
	\item 脚本模式中,4、5拨码同时向上拨启动单步调试,随后下拨上拨一次拨码5执行下一条指令。
\end{enumerate}

\section{系统使用说明}

使用说明已在功能列表中阐述。

在开发板上的实际按钮如下图所示:

\section{系统结构说明}

\section{子模块功能说明}
\subsection{人工模式}

\subsection{脚本模式}

\section{bonus 的实现说明}

\subsection{高效脚本设计}
设计理念:
\begin{enumerate}
\item 考虑到一些及其操作是全自动的,
因此操作者实际上可以利用这段时间去做一些别的事情,
以节省时间。
\item 由于操作者的移动事实上是比较耗时的,
因此在脚本中尽量避免了操作者长距离的跑动,
若是需要运输物品这样的操作,
如果能丢到附近的桌子上,
那么操作者就不会亲自跑过去一趟。
\end{enumerate}

执行完成时间:\texttt{12.6s} 左右。
\section{项目总结}

在硬件开发方面,我们团队积累了丰富的经验。通过对不同的硬件模块进行研究和开发,我们深入了解了各种硬件的特性和使用方法,并能够根据需求进行相应的选择和调整。在实际的硬件开发过程中,我们熟练掌握了各种工具和技术,并能够快速解决遇到的问题。同时我们在调错中深入理解了硬件编程与软件编程的区别,培养了硬件编程的思维。

在团队合作方面,我们团队具备良好的沟通和协作能力。通过合理的任务分配和提供清晰的接口文档,我们能够有效地协调各个组员的工作,确保整个项目顺利进行。此外,我们定期召开团队会议,及时交流项目进展和存在的问题,并寻求共同解决方案。

在开发工作中,我们使用\texttt{github}平台进行版本控制和团队协作。通过合理的分支管理和代码审查,我们能够有效地协同完成项目的不同部分,并保证代码质量和可维护性。同时我们除了使用\texttt{Vivado}进行代码编辑,还安装了\texttt{VScode}中的插件对\texttt{verilog}进行编辑。

在测试工作中,我们采用了仿真测试的方法,对每个模块进行了全面的测试,并及时修复发现的问题。通过不断完善测试流程和测试用例,我们能够保证项目的质量和稳定性。同时我们在测试代码的过程中讲变量绑到\texttt{led}灯中进行查看,这帮助我们发现了不少问题的原因。例如我们在编写\texttt{Wait}模块时,我们等待的时长一直不稳定,通过仿真测试,我们发现我们的一个变量的预期返回结果只会持续一个时钟周期,我们才修复了\texttt{bug};在编写脚本执行时,我们通过\texttt{led}灯发现\texttt{ActionScriptHandler}模块传入的\texttt{feedback}因为硬件原因位移了一位,使得\texttt{Get}指令执行不了。这两种测试方法给予了我们很大的帮助。

\section{其他的想法和建议}
\begin{enumerate}
\item 首先,可以将几个之前的项目保留,作为祖传项目,每年做一点修改然后发布,这样不至于在项目上线后还有各种\texttt{bug}和问题,以后的同学们做起来也会舒服一点。(但是这样可能就要加强一下作弊的管控,谨防抄袭往年作品)
\item 关于新题目,可以做俄罗斯方块等一些小游戏。这样可以有人实现\texttt{VGA}的接口,同时助教也可以提供\texttt{UART},(俄罗斯方块和这次厨房的\texttt{project}类似,有游戏结束和开始、旋转方块、处理不同类型方块、消除已经可以消除的方块等模块,还有对于不同方块降落下来的\texttt{feedback},以及方块下降时的等待语句等,同时还可以设计通过七段数码管显示分数、得到最高分数/以最快的时间执行完某种固定的下降模式等\texttt{bonus}),这样在加强同学们对于硬件语言的理解时也不至于过于复杂。
\end{enumerate}
\end{document}
