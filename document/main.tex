\documentclass[12pt, a4paper]{ctexart}

\usepackage{amsmath, amsthm, amssymb, graphicx,float}
\usepackage[bookmarks=true, colorlinks, citecolor=blue, linkcolor=black]{hyperref}

\title{数字逻辑项目文档:GenshinKitchen}
\author{黄政东\\祝超\\何俞均}
\date{}

\begin{document}
\maketitle

\newpage
\section{团队分工}
\subsection{团队分工}
贡献比:1:1:1\\
详细工作安排:
\begin{table}[h]
	\centering
	\begin{tabular}{|c|c|}
		\hline
		~      & project 顶层模块的处理(manualTop, ScriptTop 以及 DemoTop) \\
		\cline{2-2}
		~      & 实现接收并用LED显示来自客户端的四个反馈信号                 \\
		\cline{2-2}
		黄政东 & 所有模块的代码规范性检查                                    \\
		\cline{2-2}
		~      & 实现在手动模式/自动(脚本)模式之间进行切换                   \\
		\cline{2-2}
		~      & 实现脚本单步调试以及手动模式/自动(脚本)模式之间的切换         \\
		\cline{2-2}
		~      & 部分文档攥写工作                                            \\
		\hline

		~      & project 手动模式的合法性检查                                \\
		\cline{2-2}
		~      & project 脚本模式中jump,wait语句相关模块执行的处理          \\
		\cline{2-2}
		祝超   & 各种需要的测试脚本的准备,高效的脚本设计                    \\
		\cline{2-2}
		~      & loadingLamp 模块的编写                                      \\
		\cline{2-2}
		~      & 部分文档攥写工作                                            \\

		\hline
		~      & project 手动模式的按键到机器码部分(buttonDecoder模块)     \\
		\cline{2-2}
		~      & project 脚本模式中Action、Game State Instruction部分的处理  \\
		\cline{2-2}
		何俞均 & 使用七段数码管显示信息                   \\
		\cline{2-2}
		~      & 错误脚本状态自动处理                                        \\
		\cline{2-2}
		~      & 部分文档攥写工作                                            \\
		\hline
	\end{tabular}
	\caption{详细工作安排}
\end{table}

\subsection{开发计划日程安排和实施情况}

\section{系统功能列表}

\begin{enumerate}
	\item 1
\end{enumerate}

\section{系统使用说明}

\section{系统结构说明}

\section{子模块功能说明}
\subsection{人工模式}

\subsection{脚本模式}

\section{bonus 的实现说明}
\subsection{错误脚本自动处理}

\subsection{高效脚本设计}
设计理念:\\
1.考虑到一些及其操作是全自动的,
因此操作者实际上可以利用这段时间去做一些别的事情,
以节省时间。\\
2.由于操作者的移动事实上是比较耗时的,
因此在脚本中尽量避免了操作者长距离的跑动,
若是需要运输物品这样的操作,
如果能丢到附近的桌子上,
那么操作者就不会亲自跑过去一趟。

\section{项目总结}



\section{其他的想法和建议}
1.首先,可以将几个之前的项目保留,作为祖传项目,
每年做一点修改然后发布,
这样不至于在项目上线后还有各种bug和问题,
以后的同学们做起来也会舒服一点。
(但是这样可能就要加强一下作弊的管控,谨防抄袭往年作品)\\
2.关于新题目,可以做俄罗斯方块。
这样可以有人实现vga的接口,
同时助教也可以提供uart(就和这次厨房的proj差不多)\\




\end{document}
